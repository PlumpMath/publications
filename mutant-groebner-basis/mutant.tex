\documentclass[preprint,draft]{elsarticle}

\hyphenation{Mu-tant-XL}

\usepackage[utf8x]{inputenc}
\usepackage{graphicx}
\usepackage{color}
\usepackage{amsmath,amssymb}
\usepackage{xspace}
\usepackage{url}
\usepackage{mathptmx}

\usepackage[normalem]{ulem}

\usepackage[vlined,linesnumbered]{algorithm2e}

\newcommand{\red}[1]{\textcolor{red}{#1}}
\newcommand{\blue}[1]{\textcolor{blue}{#1}}
\newcommand{\replace}[2]{\sout{#1} {#2}}

\newcommand{\NN}{{\mathbb N}}
\newcommand{\ZZ}{{\mathbb Z}}
\newcommand{\QQ}{{\mathbb Q}}
\newcommand{\RR}{{\mathbb R}}
\newcommand{\CC}{{\mathbb C}}
\newcommand{\F}{{\mathbb F}}
\newcommand{\ie}{\textit{i.e.},\xspace}
\newcommand{\f}{\mathbf{f}}
\newcommand{\PK}{\mathbf{g}}
\newcommand{\HM}{{\rm HM}}

\newcommand{\sig}[1]{\textnormal{sig}(#1)\xspace}
\newcommand{\e}{\textbf{e}}
\newcommand{\LV}[1]{\ensuremath{\textsc{LV}(#1)\xspace}}
\newcommand{\LT}[1]{\ensuremath{\textsc{LT}(#1)\xspace}}
\newcommand{\LM}[1]{\ensuremath{\textsc{LM}(#1)\xspace}}
\newcommand{\LC}[1]{\ensuremath{\textsc{LC}(#1)\xspace}}
\newcommand{\LCM}[1]{\ensuremath{\textsc{LCM}(#1)\xspace}}

\newcommand{\GLn}{GL_{n}(\F)}
\newcommand{\GLu}{GL_{u}(\F)}
\newcommand{\Sn}{{\mathcal S}_n}

\newcommand{\FX}{\ensuremath{{{\F}[x_0,\ldots,x_{n-1}]}}}
\newcommand{\sys}{\ensuremath{f_0,\ldots,f_{m-1}}\xspace}
\newcommand{\ideal}[1]{\ensuremath{\langle #1\rangle}}

\newcommand{\malbnote}[1]{\red{malb: #1}}
\newcommand{\lpnote}[1]{\red{Ludovic: #1}}
\newcommand{\jcfnote}[1]{\red{JCF: #1}}
\newcommand{\ccnote}[1]{\red{Carlos: #1}}

\newtheorem{lemma}{Lemma}
\newtheorem{theorem}{Theorem}
\newtheorem{proposition}{Proposition}
\newdefinition{definition}{Definition}
\newdefinition{example}{Example}
\newproof{proof}{Proof}
\newtheorem{corollary}{Corollary}


\begin{document}

\title{On the relation between the MXL family of algorithms and Gr\"obner basis algorithms}
\author[lip6]{Martin R.~Albrecht}
\ead{malb@lip6.fr}
\author[rhul]{Carlos Cid} 
\ead{carlos.cid@rhul.ac.uk}
\author[lip6]{Jean-Charles Faug\`ere}
\ead{jean-charles.faugere@inria.fr}
\author[lip6]{Ludovic Perret}
\ead{ludovic.perret@lip6.fr}

\address[lip6]{
INRIA, Paris-Rocquencourt Center, POLSYS Project\\
UPMC Univ Paris 06, UMR 7606, LIP6, F-75005, Paris, France\\
CNRS, UMR 7606, LIP6, F-75005, Paris, France\\
}
\address[rhul]{Information Security Group, Royal Holloway, University of London, UK}


\begin{abstract}
The computation of Gr\"obner bases remains one of the most powerful methods for tackling the Polynomial System Solving (PoSSo) problem. 
The most efficient known algorithms reduce the Gr\"obner basis computation to Gaussian
eliminations on several matrices. However, several degrees of freedom are available to generate these matrices. It is well known that the particular strategies used can drastically affect the efficiency of the computations. 
In this work we investigate a recently-proposed strategy, the so-called {\it ``Mutant strategy"}, on which a new family of algorithms is based (MXL, MXL$_2$ and MXL$_3$).  By studying and describing the algorithms based on Gr\"obner basis concepts,
we demonstrate that the Mutant strategy can be understood to be equivalent to the classical Normal Selection strategy currently used in Gr\"obner basis algorithms.  Furthermore, we show that the ``partial enlargement'' technique can be understood as a strategy for restricting the number of S-polynomials considered in an iteration of the $F_4$ Gr\"obner basis algorithm,
while the new termination criterion used in MXL$_3$ does not lead to termination at a lower degree than the classical 
Gebauer-M\"oller installation of Buchberger's criteria.
We claim that our results map all novel concepts from the MXL family of algorithms to their well-known Gr\"obner basis equivalents. 
Using previous results that had shown the relation between the original XL algorithm and $F_4$, we conclude that  the MXL family of algorithms can be fundamentally reduced to redundant variants of $F_4$.
\end{abstract}

\begin{keyword}
Gr\"obner bases, polynomial system solving, mutants.
\end{keyword}
\maketitle 

\section{Introduction} \label{sec:intro}
The past few years have witnessed a growing interest from the cryptographic community in computational algebra methods, in particular Gr\"obner basis algorithms \cite{Buchberger65,DBLP:journals/jsc/Buchberger06a}.
This was motivated by the proposal of algebraic attacks against stream ciphers~\cite{courtois-meier-euro03} and block ciphers~\cite{courtois-pieprzyk:asiacrypt02,DBLP:conf/cisc/FaugereP09,DBLP:conf/fse/AlbrechtC09,ACTFP10}, as well as
by the proposal of several public-key schemes based on systems of multivariate polynomial equations (e.g.,~\cite{patarin-euro96}), and the corresponding cryptanalysis using the
$F_5$ algorithm~\cite{faugere-joux-crypto03,DBLP:conf/crypto/FaugereP06,DBLP:conf/crypto/FaugereLP08,DBLP:conf/pkc/BouillaguetFFP11}.  One particular algorithm has received considerable attention from the cryptographic community: the XL algorithm \cite{courtois-klimov-patarin-shamir:eurocrypt2000} 
(and its several variants, e.g.,~\cite{courtois-patarin:ct-rsa03,courtois-pieprzyk:asiacrypt02,Courtois2004}) was originally proposed by cryptographers to tackle problems arising specifically from cryptology. Although not strictly a Gr\"obner basis algorithm, it used a similar idea to the one proposed by Lazard~\cite{lazard-eurocal83}: it constructs the Macaulay matrix up to some large degree $D$ and reduces it to obtain the solution of the system. The algorithm was shown to work only under particular conditions~\cite{diem-asia04}, while other flaws were also shown in other high-profile variants~\cite{Cid2005a,DBLP:conf/fse/LimK07}.
Eventually, it was shown that the XL algorithm could be described essentially as a redundant (and less efficient) variant of the $F_4$ algorithm \cite{DBLP:conf/asiacrypt/ArsFIKS04}. That is, one can simulate the XL algorithm using a variant of the $F_4$ algorithm.

Despite of these results, because of its simplicity the XL algorithm continues to attract the attention of researchers working in cryptography~\cite{mxl,thomae-wolf:eprint2010}. In this paper we investigate a prominent recent addition to the XL family, namely the MutantXL algorithms~\cite{mxl,mxl2,mxl3,mxl4}. The concept of Mutants was first introduced in \cite{mxl}, giving rise to a family of algorithms and techniques \cite{mxl2,mxl3,mxl4}, which showed to be particularly efficient against the MQQ multivariate cryptosystem \cite{mohamed-werner-ding-buchmann:cans09}.  Unlike the XL algorithm, some of the Mutant algorithms (e.g.,~ $\textrm{MXL}_3$~\cite{mxl3}) do in fact explicitly compute the Gr\"obner basis of the corresponding ideal, assuming it is zero-dimensional.
Because of the remarkable experimental results reported in~\cite{mxl3}, a natural question arises: what is behind such a performance? Is it due to changes in the algorithm, implementation tricks, tuning towards particular problems, or perhaps a fundamentally novel algorithmic idea?

In the MutantXL literature \cite{mxl,mxl2,mxl3,mxl4} the observed performance gains are attributed to algorithmic advances. Hence, in order to compare the MutantXL family of algorithms to standard techniques in computational commutative algebra, we need to describe both in common terms. This will allow us to answer the question, whether  mutants are a new concept or whether they can be described based on well-known computational algebra concepts. Likewise, are the new {\it mutant strategies} general enough, so that they can potentially be incorporated to existent Gr\"obner basis algorithms?

There has been so far no in-depth study of the mathematical properties of mutants and related strategies, and how they are connected to other Gr\"obner basis algorithms. Because of this, there is a considerable gap between the symbolic computation and the cryptographic communities. Both investigate efficient algorithms for solving polynomial systems but results seem incommensurable in terms of strategy.

In this work, we undertake the task to bridge this gap. In particular, we compare the MXL family with two variants of the $F_4$ algorithm \cite{F4}: first, the so-called simplified $F_4$ which does not use Buchberger's criteria to avoid useless reductions to zero and second, the full $F_4$ as specified in \cite{F4}. Considering these algorithms, we show that the Mutant strategy can be understood as essentially equivalent to the Normal Selection strategy as used in Gr\"obner basis algorithms, such as $F_4$. Based on previous results, which showed the relation between the XL algorithm and $F_4$ \cite{DBLP:conf/asiacrypt/ArsFIKS04}, we conclude that MXL can too be described as a redundant variant of $F_4$. Furthermore, we also study the ``partial enlargement'' strategy proposed in \cite{mxl2} and demonstrate that it corresponds to selecting a subset of S-polynomials in Gr\"obner basis algorithms. As a result, we conclude that MXL$_2$ can also be described as a variant of $F_4$, although a variant that diverges from known approaches about how to select the number of S-polynomials in each iteration.  Finally, we consider the new termination criterion proposed in \cite{mxl3} and demonstrate that it does not lead to a lower degree of termination than using Buchberger's criteria to remove useless pairs in a Gröbner basis algorithm. As a result, we reach the conclusion that MXL$_3$ can  be reduced to a redundant variant of the full $F_4$ algorithm.

Our work is in the tradition of previous papers comparing different approaches for polynomial system solving \cite{Mandache-phd,
Mandache-tc,Mandache1994}. We stress, however, that the equivalence of algorithms presented in this work is {\em constructive}, i.e., we show that the Mutant family of algorithms can be simulated using redundant variants of the $F_4$ algorithm.

The remaining of this work is organised as follows.  In Section~\ref{sec:xl} we recall the well-known XL algorithm, and re-state the result showing the relation between XL and $F_4$.  In Section~\ref{sec:gb} we review well-known statements from commutative algebra. For the sake of exposition, we place particular emphasis on the concept of
S-polynomials and the central role they play in Gr\"obner bases computations. In particular, we show that in XL-style algorithms any multiplication of polynomials by monomials except for those giving rise to S-polynomials is redundant. In Section~\ref{sec:mutants} we review the definition of Mutants, and present our pseudocode for the MXL$_3$ algorithm. In Section~\ref{sec:relation} we state and prove our main result, namely that the Mutant strategy is a redundant variant of the Normal Selection strategy. We also treat partial enlargement and the termination condition of MXL$_3$ in Section~\ref{sec:relation}. We conclude in Section~\ref{sec:conclusion}, where we include a brief discussion on what we view as the limitations of using running times as the {\it sole} basis for comparison between Gr\"obner basis {\it algorithms}.


\section{The XL Algorithm} \label{sec:xl}
In this section we briefly recall the well-known XL algorithm.
An iterative variant of the algorithm is given in Algorithm~\ref{alg:xl}. We adopt the notation from \cite{mxl3} and,
given a set of polynomials $S$, we denote by $S_{(op)d}$ the subset of $S$ with elements of degree $(op)d$ where $(op) \in \{=,<,\leq,>,\geq\}$.

\begin{algorithm}
\KwIn{$F$ -- a tuple of polynomials}
\KwIn{$D$ -- an integer $> 0$} 
\KwResult{a $D$-Gröbner basis for $F$} 
\SetKw{KwContinue}{continue}
\Begin{
$G \longleftarrow \varnothing$\;
\For{$1 \leq d \leq D$}{
  $F_{=d} \longleftarrow \varnothing$\;
  
  \For{$f \in F$}{
    \uIf{$\deg(f) = d$}{add $f$ to $F_{=d}$\;}
    \ElseIf{$\deg(f) < d$}{
      $M_{=d-\deg(f)} \longleftarrow$ all monomials of degree $d-\deg(f)$\;
      \For{$m \in M_{=d-\deg(f)}$}{
        add $m\cdot f$ to $F_{=d}$\;
      }
    }
  }
  $G \longleftarrow$ the row echelon form (of the matrix) of $G\ \cup F_{=d}$\;
 }
\Return{$G$}
}
\caption{XL \label{alg:xl}}
\end{algorithm}

It was shown in \cite{DBLP:conf/asiacrypt/ArsFIKS04} that the XL algorithm can be emulated using the $F_4$ algorithm.
In particular, \cite{DBLP:conf/asiacrypt/ArsFIKS04} proves that:
\begin{lemma}
\label{lem:xl}
XL (described in Algorithm~\ref{alg:xl}) can be simulated using  $F_4$  (described in Algorithm~\ref{alg:f4}) by adding redundant pairs.
\end{lemma}

\noindent
A simple corollary of this result is that the following holds when both algorithms only compute up to a fixed degree $D$.
\begin{corollary}
\label{corollary:iteration}
Let $G_{XL,D}$ be the set of polynomials computed by the XL algorithm up to degree $D$.
Then  $\forall g \in G_{XL,D}$, there exists $f \in G_{F4,D}$ with $\LM{f} \mid \LM{g}$,  where $G_{F4,D}$ is the set of polynomials computed by the $F_4$ algorithm up to degree $D$.
\end{corollary}



\section{Gr\"obner Bases Basics} \label{sec:gb}
In this section we recall some basic results about Gr\"obner bases. For a more detailed treatment, we refer the reader to, for instance, \cite{Cox1992}. Consider a polynomial ring $R = \F[x_0,\dots,x_{n-1}]$ over some finite field $\F$. We adopt some admissible ordering on monomials in $R$. We can then denote by $\LM{f}$ the largest or leading monomial appearing in $f \in R$ and by $\LC{f} \in \F$ the coefficient corresponding to $\LM{f}$ in $f$. By $\LT{f}$ we denote $\LC{f} \cdot \LM{f}$. In this work $\LV{f}$ denotes the largest variable -- ordered w.r.t. the monomial ordering -- in the leading monomial $\LM{f}$ of $f$, 
and given a set $F \subset R$, we define
$\LV{F,x}$ as $\{f \in F \mid \LV{f} = x\}$. 
The set of leading monomials of $F$ is defined as $\LM{F} = \{\LM{f} \mid f \in F\}$, $M$ denotes the set of all monomials in $R$, 
while $M(F)$ is the set of all monomials appearing in the polynomials in $F$. 

The ideal $\mathcal{I}$ generated by \sys $\in R$, denoted $\ideal{\sys}$, is defined 
as $$\left\{ \sum_{i=0}^{m-1} h_i f_i  \mid h_0 ,\dots , h_{m-1} \in R \right\}.$$ It is well-known that every ideal  $\mathcal{I} \subseteq R$ is finitely generated.
A Gr\"obner basis of an ideal $\mathcal{I}$ is a particular set of generators. 
\begin{definition}[Gr\"obner Basis]
Let $\mathcal{I}$ be an ideal of $\F[x_0,\dots,x_{n-1}]$ and fix a monomial ordering. A finite subset $$G = \{g_0 ,\dots , g_{m-1} \} \subset \mathcal{I}$$  is said to be a \emph{Gr\"obner basis} of $\mathcal{I}$ if for any $f \in \mathcal{I}$ there exists $g_i \in G$ such that $\LM{g_i} \mid \LM{f}$.
\end{definition}
We note that if a set of polynomials $\sys$ has a unique root, i.e. the system of equations $f_0=0,\ldots,f_{m-1}=0$ has a unique solution, then computation of the Gr\"obner basis of the corresponding
ideal allows one to solve the system (i.e. the solution can be ``read" directly on the Gr\"obner basis).
More generally, if the ideal is zero-dimensional, the solutions of a system can be computed from a Gr\"obner basis in polynomial-time (in the number of solutions) \cite{FGLM}.
  
Since the notion of Gr\"obner bases is defined by the existence of {\it relatively} low leading terms, the task of computing a Gr\"obner basis is essentially to find new elements in the ideal with lower leading terms until no more such elements can be found. Buchberger proved in his PhD thesis~\cite{Buchberger65} that Gr\"obner bases can be computed by considering only S-polynomials. Such polynomials are designed to cancel leading terms and thus potentially produce new elements in the ideal with lower leading terms.
\begin{definition}[S-Polynomial]
\label{def:spolynomials}
Let $f,g\ \in\ \F[x_0,\dots,x_{n-1}]$ be non-zero polynomials.
\begin{itemize}
\item Let $\LM{f} = \prod_{i=0}^{n-1} x_i^{\alpha_i}$ and $\LM{g} = \prod_{i=0}^{n-1} x_i^{\beta_i}$, with $\alpha_i,\beta_i \in \NN$, denote the leading monomials
of $f$ and $g$ respectively.
Set $\gamma_i = \max(\alpha_i,\beta_i)$ for every $0 \leq  i < n$, and denote by $x^\gamma\ = \prod_{i=0}^{n-1} x_i^{\gamma_i}$.  
It holds that $x^\gamma$ is the least common multiple of
$\LM{f}$ and $\LM{g}$, written as $$x^\gamma\ =\LCM{\LM{f},\LM{g}}.$$  
\item The {\it S-polynomial} of $f$ and $g$ is defined as
$$
S(f,g)\ =\ \frac{x^\gamma}{\textsc{LT}(f)}\cdot f\ -\
\frac{x^\gamma}{\textsc{LT}(g)}\cdot g.
$$
\end{itemize}
\end{definition}

Now let $G=\{g_0,\ldots, g_{s-1}\} \subset R$, and $\mathcal{I}$ be the ideal generated by $G$.
We say that a polynomial $f \in \mathcal{I}$ has a {\it standard representation} w.r.t. $G$ if 
there exist constants $a_0,\ldots, a_{s-1} \in \F$ and monomials $t_0,\ldots, t_{s-1} \in M$ such that     
$$
f=\sum_{k=0}^{s-1}a_k t_k g_k, 
$$    
with $\LM{t_k g_k} \leq \LM{f}$. Buchberger's main result stated that $G$ is a Gr\"obner basis for $\mathcal{I}$ if and only if every 
S-polynomial $S(g_i,g_j)$ has a  {\it standard representation} w.r.t. $G$.

Furthermore, Buchberger showed that in the computation of Gr\"obner bases it is \emph{sufficient} to consider S-poly\-nomials only, since \emph{any }reduction of leading terms can be attributed to S-polynomials. 
There are many variants of this result in textbooks on commutative algebra; we give below the statement and proof based on \cite{Cox1992} since the presentation helps to understand the close connection between 
XL and Gr\"obner basis algorithms.
The proof is included for the sake of completeness.
\begin{lemma} \label{lemma:cancel}
Let $f_0,\ldots,f_{t-1}$ be nonzero polynomials in $R$. 
Given a monomial $x^{\delta}$ such that $\LM{f_i} \mid x^{\delta}$ for all $i=0,\ldots,t-1$, let $x^{\alpha(0)}$, $\ldots$,  $x^{\alpha(t-1)}$ be monomials in $R$ such that \(x^{\alpha(i)}\,\LM{f_{i}}=x^{\delta}\) for all \(i\).
We consider the sum \(f = \Sigma^{t-1}_{i=0} c_ix^{\alpha(i)}f_i\), where \(c_{0},\ldots,c_{t-1}\in \mathbb{F}\backslash\{0\}\). If \(\LM{f}<x^{\delta}\), then there exist constants $b_{j} \in \F$ such that
\begin{eqnarray} \label{wegheben} 
f=\sum^{t-1}_{i=0} c_ix^{\alpha(i)}f_i\ &=&\ \sum^{t-2}_{j=0}  b_{j}x^{\delta-\tau_{j}}\, S(f_{j},f_{j+1}),
\end{eqnarray}
where \(x^{\tau_{j}} = \LCM{\LM{f_{j}},\LM{f_{j+1}}}\). Furthermore
$$
x^{\delta-\tau_{j}}S(f_{j},f_{j+1}) < x^\delta, \mbox{ for all } j=0,\ldots,t-2.
$$
\end{lemma}
\begin{proof}
Let $d_i = \LC{f_i}$. It follows that $c_id_i$ is the leading coefficient of $c_ix^{\alpha(i)}f_i$.  Furthermore, let $p_i = \frac{x^{\alpha(i)}f_i}{d_i}$ and thus $\LC{p_i} = 1$. Consider the ``telescope sum'':
\begin{eqnarray*}
f &=&\sum^{t-1}_{i=0} \ c_ix^{\alpha(i)}f_i=\sum^{t-1}_{i=0}\ c_i d_i\dfrac{x^{\alpha(i)}f_i}{d_i}=\sum^{t-1}_{i=0}\ c_id_ip_i\\
&=&\sum^{t-1}_{i=0}\left( \sum_{j=0}^{i}c_{j}d_{j} - \sum_{j=0}^{i-1}c_{j}d_{j} \right)\, p_{i}\\
&=&\sum^{t-1}_{i=0} \sum_{j=0}^{i}c_{j}d_{j}p_{i} -\sum^{t-2}_{i=-1} \sum_{j=0}^{i}c_{j}d_{j} \, p_{i+1}\\
&=&\sum_{j=0}^{t-1}c_{j}d_{j}p_{t-1}+\sum^{t-2}_{i=0} \sum_{j=0}^{i}c_{j}d_{j}(p_{i} - p_{i+1}).\\
\end{eqnarray*}
All $c_ix^{\alpha(i)}f_i$ have $x^\delta$ as leading monomial. Since their sum has smaller leading monomial, we have that 
$\Sigma^{t-1}_{i=0}c_id_i = 0$, leading to:
\begin{eqnarray} \label{teleskop}
f = \sum^{t-2}_{i=0} \sum_{j=0}^{i}c_{j}d_{j}(p_{i} - p_{i+1}).
\end{eqnarray}
By assumption  \(x^{\alpha(i)}\,\LM{f_{i}}=x^{\delta}\) for all \(i=0,\ldots,t-1\), and we have:
\begin{eqnarray*}
x^{\delta - \tau_{j}}S(f_j,f_{j+1}) &=& x^{\delta - \tau_{j}} \left(\dfrac{x^{\tau_{j}}}{\LT{f_j}}f_j - \dfrac{x^{\tau_{j}}}{\LT{f_{j+1}}}f_{j+1}\right)\\
&=& \dfrac{x^{\alpha(j)}}{d_j}f_j - \dfrac{x^{\alpha(j+1)}}{d_{j+1}}f_{j+1}\\
&=& p_j - p_{j+1}.
\end{eqnarray*}
This is now plugged into the telescope sum \eqref{teleskop} leading to:
\[
f = \sum^{t-2}_{i=0} \sum_{j=0}^{i}c_{j}d_{j}x^{\delta - \tau_{i}}S(f_i,f_{i+1})
 = \sum^{t-2}_{i=0} b_{i}  x^{\delta - \tau_{i}}  S(f_{i},f_{i+1}),
\]
with \(b_{i}=\sum_{j=0}^{i}c_{j}d_{j}\).
Since the polynomials $p_j$ and $p_{j+1}$ have leading monomial $x^\delta$ and leading coefficient $1$, the difference $p_j\ -\ p_{j+1}$ has a smaller leading monomial. Since we have that $p_j\ -\ p_{j+1}\ =\  x^{\delta - \tau_{j}}S(f_j, f_{j+1})$, this claim also holds true for $x^{\delta - \tau_{j}}S(f_j,f_{j+1})$. Thus the Lemma holds. \qed
\end{proof}


   
The following corollary is a simple generalisation of Lemma~\ref{lemma:cancel} to sums where not all summands have the same leading term.
\begin{corollary}
\label{corollary:cancel_general}
Let $f_0,\ldots,f_{t-1}$ be polynomials in $R$. 
Consider the polynomial $f$ as the sum $f = \Sigma^{t-1}_{i=0} c_ix^{\alpha(i)}f_i$, with coefficients 
$c_0, \ldots, c_{t-1} \in \mathbb{F}\backslash\{0\}$,
such that $\LM{f} < x^\delta = \max\{x^{\alpha(i)}\LM{f_i}\}$. 
Without loss of generality, we can assume that there is a $\tilde{t}$ such that $x^{\alpha(j)}\LM{f_j} = x^\delta$ for $j < \tilde{t}$ and $x^{\alpha(k)}\LM{f_k} < x^\delta$ for $k \geq \tilde{t}$.  
Then there exist constants $b_{i} \in \F$ such that
\begin{eqnarray*}
f &=& \sum_{i=0}^{\tilde{t}-2} b_{i}x^{\delta-\tau_{i}}S(f_i,f_{i+1}) + \sum_{k=\tilde{t}}^{t-1} c_kx^{\alpha(k)}f_k\\
 &=& \sum \tilde{c_i}x^{\tilde{\alpha}(i)}\tilde{f}_i,
\end{eqnarray*}
where \(x^{\tau_{j}} = \LCM{\LM{f_{j}},\LM{f_{j+1}}}\), $\tilde{c_i} x^{\tilde{\alpha}(i)} \tilde{f_i} = c_{i+1} x^{\alpha(i+1)} f_{i+1}$ if $i\geq \tilde{t}-1$ and $b_{i}x^{\delta-\tau_{i}}S(f_i,f_{i+1})$ otherwise. Furthermore, for all $0 \leq i \leq \tilde{t}-2$, we have $$\LM{x^{\delta-\tau_{i}}S(f_i,f_{i+1})} < x^\delta$$ and thus 
$$
x^{\tilde{\alpha}(i)}\LM{\tilde{f}_i} < x^\delta \mbox{  for all } i.
$$
\end{corollary}

Corollary~\ref{corollary:cancel_general} states essentially that whatever cancellations can be produced by monomial multiplications and $\F$-linear combinations, they can be attributed to S-polynomials. It follows that 
the only cancellations that need to be considered in an XL-style algorithm are those produced by S-polynomials. 

\begin{example}
Consider the polynomials $f =xy + x + 1$, $g = x + 1$ and $h = z + 1 \in \F_{127}[x,y,z]$, a pathological example constructed to demonstrate the role of S-polynomials. We fix the degree reverse lexicographical term ordering. To compute a Gr\"obner basis, we start by constructing two S-polynomials of degree two, namely: $f - y\cdot g = x - y + 1$ and $z\cdot g - x\cdot h = -x + z$. We note that the latter trivially reduces to zero and would be detected and avoided by Buchberger's first criterion \cite{Cox1992}. Ignoring this optimisation, in matrix notation, we would have to consider the six rows corresponding to $f, y\cdot g, z\cdot g, x\cdot h, g$ and $h$.
For comparison, XL would consider the following polynomials up to degree two.
\begin{center}
\begin{tabular}{rrrrrr}
$f  =$ & $xy + x + 1$, & \ \ $x\cdot g$ = & $x^2 + x$, & \ \ $y\cdot g =$ & $xy + y$,\\
$z\cdot g$ = & $xz + z$,& \ \   $x\cdot h =$ & $xz + x$,    & \ \ $y\cdot h$ = & $yz + y$,\\
$z\cdot h =$ & $z^2 + z$,    & \ \ $g $ = & $x + 1$, & \ \ $h  =$ & $z + 1$.\\
\end{tabular}
\end{center}

In matrix notation we have
\begin{scriptsize}
\[
A = \left(\begin{array}{rrrrrrrrr}
0 & 1 & 0 & 0 & 0 & 1 & 0 & 0 & 1 \\
1 & 0 & 0 & 0 & 0 & 1 & 0 & 0 & 0 \\
0 & 1 & 0 & 0 & 0 & 0 & 1 & 0 & 0 \\
0 & 0 & 1 & 0 & 0 & 0 & 0 & 1 & 0 \\
0 & 0 & 1 & 0 & 0 & 1 & 0 & 0 & 0 \\
0 & 0 & 0 & 1 & 0 & 0 & 1 & 0 & 0 \\
0 & 0 & 0 & 0 & 1 & 0 & 0 & 1 & 0 \\
0 & 0 & 0 & 0 & 0 & 1 & 0 & 0 & 1 \\
0 & 0 & 0 & 0 & 0 & 0 & 0 & 1 & 1
\end{array}\right) \mbox{  and  }
E = \left(\begin{array}{rrrrrrrrr}
1 & 0 & 0 & 0 & 0 & 0 & 0 & 0 & -1 \\
0 & 1 & 0 & 0 & 0 & 0 & 0 & 0 & 0 \\
0 & 0 & 1 & 0 & 0 & 0 & 0 & 0 & -1 \\
0 & 0 & 0 & 1 & 0 & 0 & 0 & 0 & 0 \\
0 & 0 & 0 & 0 & 1 & 0 & 0 & 0 & -1 \\
0 & 0 & 0 & 0 & 0 & 1 & 0 & 0 & 1 \\
0 & 0 & 0 & 0 & 0 & 0 & 1 & 0 & 0 \\
0 & 0 & 0 & 0 & 0 & 0 & 0 & 1 & 1 \\
0 & 0 & 0 & 0 & 0 & 0 & 0 & 0 & 0
\end{array}\right).\]
\end{scriptsize}
Of course, the system $f=0,g=0,h=0$ is straightforwardly solved by evaluating $f$ at $x= -1$ as implied by $g$. We note, however, that this computation is equivalent to reducing the S-polynomial $S(f,g)$ by $g$, i.e., the first step in Buchberger's algorithm.
\end{example}
Note that Lemma~\ref{lemma:cancel} does not state that $\LM{f} = \max\{\LM{S(f_j,f_{j+1})}\}$, but rather that the leading terms of summands decrease once rewritten using S-polynomials. In the following example, we consider the case when $\LM{f} < \max\{\LM{S(f_j,f_{j+1})}\}$. In this case, we can reapply Lemma~\ref{lemma:cancel} to $f_i' = S(f_i,f_j)$ as the following example emphasizes.

\begin{example} Consider the polynomials $f = xy + a$, $g = yz + b$, and $h = ab + 1$ in the polynomial ring $\F_{127}[x,y,z,a,b]$. We consider the degree reverse lexicographical term ordering.
There are three possible S-polynomials $S(f,g),S(f,h)$ and $S(g,h)$. Two of them -- $S(f,h)$ and $S(g,h)$ -- trivially reduce to zero and would be detected and avoided by Buchberger's first criterion. However, one S-polynomial does not reduce to zero: $s_0 = z\cdot f - x\cdot g = za - xb$.
From $s_0$ we can then construct $s_1 = b\cdot s_0 - z\cdot h = -xb^2 - z$, among others, also at degree 3, which is an element of the reduced Gr\"obner basis. The XL algorithm at degree 3 will produce $$\{m\cdot p \mid m \in \{1,x,y,z,a,b\}, p \in \{f,g,h\}\},$$ which reduces to 
\[
\begin{array}{llll}
 x^{2} y + x a, &\ x y^{2} + y a, &\  x y z + x b,  &\  y^{2} z + y b, \\ 
 y z^{2} + z b, &\ x y a + a^{2}, &\  y z a - 1,    &\  x y b - 1,     \\
 y z b + b^{2}, &\ x a b + x,     &\  y a b + y,    &\  z a b + z,     \\
 a^{2} b + a,   &\ a b^{2} + b,   &\  x y + a,      &\  y z + b,       \\
 z a - x b,     & \textnormal{ and } & a b + 1 &\\
\end{array}
\]
by Gaussian elimination. Note that $xb^2 + z$ is not in that list. However, if we increase the degree of XL to 4, the list returned is 
\[
\begin{array}{llll}

 x^{3} y + x^{2} a, &\ x^{2} y^{2} - a^{2}, &\ x y^{3} + y^{2} a, &\ x^{2} y z + x^{2} b,\\
 x y^{2} z + 1,     &\ y^{3} z + y^{2} b,   &\ x y z^{2} + x z b, &\ y^{2} z^{2} - b^{2},\\
 y z^{3} + z^{2} b, &\ x^{2} y a + x a^{2}, &\ x y^{2} a + y a^{2},&\ x y z a - x,\\
 y^{2} z a - y,     &\ y z^{2} a - z,       &\ x y a^{2} + a^{3}, &\ y z a^{2} - a,\\
 x^{2} y b - x,     &\ x y^{2} b - y,       &\ x y z b - z,  &\ y^{2} z b + y b^{2},\\
 y z^{2} b + z b^{2}, &\ x^{2} a b + x^{2}, &\ x y a b - a, &\ y^{2} a b + y^{2},\\
 x z a b + x z. &\ y z a b - b,             &\ z^{2} a b + z^{2},&\ x a^{2} b + x a,\\
 y a^{2} b + y a, &\  z a^{2} b + x b, &\  a^{3} b + a^{2}, &\  x y b^{2} - b,\\
 y z b^{2} + b^{3}, &\  x a b^{2} + x b, &\  y a b^{2} + y b, &\  z a b^{2} + z b, \\ 
 a^{2} b^{2} - 1, &\  a b^{3} + b^{2}, &\  x^{2} y + x a, &\  x y^{2} + y a, \\ 
 x y z + x b, &\  y^{2} z + y b, &\  y z^{2} + z b, &\  x y a + a^{2}, \\ 
 x z a - x^{2} b, &\  y z a - 1, &\  z^{2} a - x z b, &\  z a^{2} + x, \\ 
 x y b - 1, &\  y z b + b^{2}, &\  x a b + x, &\  y a b + y, \\ 
 z a b + z, &\  a^{2} b + a, &\  {\bf x b^{2} + z}, \ & a b^{2} + b, \\ 
 x y + a, &\  y z + b, &\  z a - x b & \textnormal{ and }  a b + 1 ,\\
\end{array}
\]
which does contain $xb^{2} + z$. Thus, XL did produce $x b^{2} + z$ in one step at degree $4$ but it could not produce $xb^2 + z$ at degree 3 since this element corresponds to $$b\cdot (z\cdot f - x\cdot g) - z\cdot h = (bz)\cdot f -(bx)\cdot g - z\cdot h,$$ but we have that $\deg(bz\cdot f) = 4$. We note that this behaviour of XL was the motivation for the Mutant concept.
\end{example}


\section{Mutants and MXL algorithms} \label{sec:mutants}    
Let $F = \{\sys \} \subset \FX$, and $\mathcal{I}=\ideal{\sys}$ be the ideal generated by $F$.
Recall that any element $f \in \mathcal{I}$ can be written as
$$f=\sum_{i=0}^{m-1}h_i \cdot f_i, \mbox{ with } h_i \in  \FX.$$
Note that this representation is usually not unique. Following the terminology of \cite{mxl}, we call the {\it level} of the representation $\sum_{f_i \in F}h_i \cdot f_i$ of $f$ the maximum degree of $\{h_i \cdot f_i \mid f_i \in F\}$. We call the {\it level} of $f$ the minimal level of all its representations. We can then define the concept of a {\it mutant}~\cite{mxl,mxl2,mxl3}.
\begin{definition}
Given a set of generators $F$ of an ideal $\mathcal{I}$, a polynomial $f \in \mathcal{I}$ is a  {\rm mutant} if its total degree is strictly less than its level.       
\end{definition}
A mutant corresponds to a ``low-degree" relation occurring during XL or more generally during any Gr\"obner basis computation. 
It follows from the discussion in Section~\ref{sec:gb} that, in the language of commutative algebra, 
a mutant occurs when an S-polynomial has a lower-degree leading monomial after reduction by $F$ and if this new leading monomial was not in the set $\LM{F}$ before reduction.

The concept of mutant has recently motivated the proposal of a family XL-style algorithms~\cite{mxl,mxl2,mxl3,mxl4}. We discuss below the most prominent, namely the MXL$_3$ algorithm.

\subsection{MXL$_3$ Algorithm} \label{sec:mxl3}
The MXL family of algorithms improves the XL algorithm using the mutant concept. In particular, the MXL$_3$ (Algorithmm~\ref{alg:mutantxl3}) differs from XL in the following respects:
\begin{enumerate}
 \item Instead of ``blindly'' increasing the degree in each iteration of the algorithm, the MXL algorithms treat mutants at the lowest possible degree, (cf.\ line \ref{mxl3:mutants_found} in Algorithm~\ref{alg:mutantxl3}). This is the key contribution of the MXL algorithm \cite{mxl}.
 \item Instead of considering all elements $F_{=d}$ of the current degree $d$, MXL$_3$ only considers a subset of elements per iteration. It incrementally adds more elements of the current degree, if the elements of the previous iteration did not suffice to solve the system (cf.\ lines \ref{mxl3:partial_enlargement}-\ref{mxl3:partial_enlargement:end} in Algorithm~\ref{alg:mutantxl3}). This is called {\it partial enlargement} in \cite{mxl2,mxl3}. This is the key contribution of the MXL$_2$ algorithm \cite{mxl2}.
 \item XL terminates at the user-provided degree $D$, while MXL$_3$ does not require to fix the degree a priori. Instead, the algorithm will terminate once a Gröbner basis was found using a new criterion (cf.\ line \ref{mxl3:terminate} in Algorithm~\ref{alg:mutantxl3}). This is the key contribution of the MXL$_3$ algorithm \cite{mxl3}.
\end{enumerate}
The pseudocode presented in Algorithm~\ref{alg:mutantxl3} is a slightly simplified variant of the MXL$_3$ algorithm; we use this presentation in Section~\ref{sec:relation} to compare it with the $F_4$ algorithm (Algorithm~\ref{alg:f4}).


\begin{algorithm}[ht]
\caption{MXL$_3$ (simplified)} 
\label{alg:mutantxl3}
\KwIn{$F$ -- a list of polynomials \sys $\in \FX$ spanning a zero-dimensional ideal.}
\KwResult{A Gr\"obner basis for $\ideal{\sys}$.}
\SetKw{KwAnd}{and}
\SetKw{KwOr}{or}
\Begin{
$D \longleftarrow \max\{\deg(f) \mid f \in F\}$\;
$d \longleftarrow \min\{\deg(f) \mid  f \in F\}$\;
$Mu \longleftarrow \varnothing$;
$newExtend \longleftarrow True$;
$x \longleftarrow x_0$;
$CL \longleftarrow d$\;
\While{True}{
  $\tilde{F}_{\leq d} \longleftarrow $ the row echelon form (or  matrix form) of $F_{\leq d}$\; 
  $Mu \longleftarrow Mu \cup \{f \in \tilde{F}_{\leq d} \mid \deg(f) < d\ \KwAnd\ \LM{f} \not\in \LM{F_{\leq d}}\}$\;  
  $F_{\leq d} \longleftarrow \tilde{F}_{\leq d}$\;
  \tcp{did we find mutants?}
  \eIf{$Mu \neq \varnothing$} {\nllabel{mxl3:mutants_found}
     $k \longleftarrow \min\{\deg(f) \mid  f \in Mu\}$\;
     $y \longleftarrow \max\{\LV{f}  \mid  f \in F_{\leq k+1}\}$\;\nllabel{mxl3:ychoice}
     $Mu_{=k}^+ \longleftarrow $ Multiply all elements of $Mu_{=k}$ by all variables $\leq y$\;
     $Mu \longleftarrow Mu \setminus Mu_{=k}$\;
     $F \longleftarrow F \cup Mu_{=k}^+$\;
     $d \longleftarrow k + 1$\;
  }{
  \tcp{does the basis contain all monomials of some degree $d_t$?}
  \If{$d < CL$ \KwAnd $M_{=d_t} \subseteq \LM{F}$ for some $1 \leq d_t \leq d$}{
    \tcp{We found a Gr\"obner basis}
    \Return{$F$}\; \nllabel{mxl3:terminate}
  }
  \tcp{did we do all enlargements at this degree already?}
  \eIf{$newExtend = True$}{
   $D \longleftarrow D + 1$\;
   $x \longleftarrow \min\{\LV{f} \mid f \in F_{=D-1}\}$\;
   $newExtend \longleftarrow False$\;
  }{
   \tcp{do partial enlargement and eliminate} 
   $x \longleftarrow \min\{\LV{f} \mid f \in F_{=D-1}\ \KwAnd\ \LV{f} > x\}$\; \nllabel{mxl3:partial_enlargement}
   $F^+ \longleftarrow $ Multiply all elements of $\LV{F,x}$ by all variables \sout{$\leq x$} without redundancies\;\nllabel{mxl3:incomplete}
   $F \longleftarrow F \cup F^+$\; \nllabel{mxl3:partial_enlargement:end}
   \If{$x= x_0$}{
    $newExtend \longleftarrow True$\;
    $CL \longleftarrow D$\;
   }
  }
  $d \longleftarrow D$\;
 }
}
}
\end{algorithm}
Our pseudocode has some minor differences with the pseudocode presented in \cite{mxl3}; we list these below:
\begin{description}
\item[Partial enlargement.] We disregard any partial enlargement strategy in the case when mutants were found. This matches the pseudocode in \cite{mxl3}. However, the actual implementation of MXL$_3$ does indeed use the partial enlargement when $Mu\neq \varnothing$ (i.e. mutants exist) \cite{mxl3-implementation}. We note that our pseudocode and that in \cite{mxl} are equivalent to MXL \cite{mxl} in this case. Since our work is mainly concerned with the concept of mutants, maintaining this simplification seems appropriate.

\item[Choice of $y$.] In line \ref{mxl3:ychoice} we set $y$ to  $\max\{\LV{f}  \mid  f \in F_{\leq k+1}\}$ instead of $\max\{\LV{f}  \mid  f \in Mu_{=k}\}$ since this allows reductions among all elements of degree $k+1$ instead of only those in $Mu_{=k+1}$. Restricting reduction to the elements of $Mu_{=k+1}$ could lead to incomplete reductions and thus results. The actual implementation of MXL$_3$ uses ``partial enlargement'' in this step and thus increases $y$ iteratively \cite{mxl3-implementation}.

\item[Incomplete reductions.] In line \ref{mxl3:incomplete} we removed the optimisation that only variables $\leq x$ are used for multiplication in the extension step. This optimisation can lead to an incorrect result as some reductions are never performed. As an example, consider $f = ab + 1$, $g = bc + a + b$ and $h = c$. The reduced Gr\"obner basis of the ideal $\ideal{f,g,h}$ over $\F_2[a,b,c]$ with respect to a degree lexicographical term ordering is $\{a + 1, b + 1, c\}$. However, the pseudocode of MXL$_3$ as described in \cite{mxl3} will not perform the necessary reductions. The leading variable of $h$ is $c$, thus $h \in \LV{F,c}$ and $h$ is never extended using any variable except $c$, since $a>c$ and $b>c$. 

Furthermore, the S-polynomial $S(f,g) = c\cdot f - a\cdot g = (abc + c) - (abc + ab + a) = ab + a + c$ is not constructed since $ag$ requires multiplication of $g$ in $\LV{F,b}$ by $a$ but $a>b$. Thus, on termination the output of MXL$_3$ is not a Gr\"obner basis. 

Our change matches Proposition 3 from \cite{mxl3}, which requires that for $H \longleftarrow \{t\cdot g \mid g \in G, t \textnormal{ a term and } \deg(t\cdot g) \leq D +1\}$ the reduced row echelon form of $H$ is $G$. However, this property is not enforced by MXL$_3$ as presented in pseudocode in \cite{mxl3}, since some $t \cdot g$ are prohibited from being constructed if $\deg(t) = 1$ and $t > \LV{g}$. We confirmed with the authors of \cite{mxl3} that their implementation catches up on those missing multiplications when $newExtend = True$ \cite{mxl3-implementation}.
\end{description}

We also present a simplified version of the $F_4$ algorithm in Algorithm~\ref{alg:f4}. For this, we need however to introduce the required notation.
 \begin{definition}
  Let $F  \subset  \F[x_0,\dots,x_{n-1}]$, and $(f,g) \in F \times F$ with $f \not =g$. We denote:
 $$
 \textsc{Pair}(f,g)=\big(\LCM{\LM{f},\LM{g}}, m_f,f,m_g,g\big),
 $$
 where $\LCM{\LM{f},\LM{g}}=\LM{m_g \cdot g}=\LM{m_f \cdot  f}$.
Now, let $P=\{\textsc{Pair}(f,g) \mid \forall (f,g) \in P \times P \mbox{ with } g > f\}, \textsc{p}=\textsc{Pair}(f,g) \in P$. We define \textsc{Left} and \textsc{Right} as:
\[
\begin{array}{cc}
    \textsc{Left}(\textsc{p})  =  (m_f,f)  &  \textsc{Right}(\textsc{p})  =  (m_g,g) ,\\
    & \\
    \textsc{Left}(P) = \bigcup_{\textsc{p} \in P} \textsc{Left}(\textsc{p}) &  \, \, \, \, \textsc{Right}(P) = \bigcup_{\textsc{p} \in P} \textsc{Right}(\textsc{p}).
 \end{array}
\]
 \end{definition}

\begin{algorithm}[ht]
\KwIn{$F$ -- a tuple of polynomials $f_0,\dots,f_{m-1}$}
\KwIn{$\textsc{Sel}$ -- a selection strategy}
\KwResult{a Gr\"obner basis for $F$}
\SetKw{Kwst}{such that}
\SetKw{KwAnd}{and}
\SetKw{KwWith}{with}
\Begin{
$G,i \longleftarrow F,0$\;
$\tilde{F}^+_i \longleftarrow F$\;
$P \longleftarrow \{\textsc{Pair}(f,g) \mid \forall f,g \in G$ \KwWith $g > f\}$\;
\While{$P \neq \varnothing$}{
  $i \longleftarrow i + 1$\;
  $P_i \longleftarrow$ $\textsc{Sel}(P)$\;
  $P \longleftarrow P\setminus P_i$\;
  $\mathcal{L}_i \longleftarrow$ Left($P_i$) $\bigcup$ Right($P_i$)\;
  \tcp{Symbolic Preprocessing}
  $F_i \longleftarrow \{t\cdot f \mid \forall (t,f) \in \mathcal{L}_i\}$\;
  $Done \longleftarrow \LM{F_i}$\;
  \While{$M(F) \neq Done$}{
    $m \longleftarrow$ an element in $M(F) \setminus Done$\;
    add $m$ to $Done$\;
    \If{$\exists\ g \in G \, \Kwst \, \LM{g}\ \mid \ m$}{
      $u = m/\LM{g}$\;
      add $u \cdot g$ to $F_i$\;
   }
  }
  \tcp{Gaussian Elimination}
  $\tilde{F}_i \longleftarrow $ the row echelon form of $F_i$\;
  $\tilde{F}_i^+ \longleftarrow \{f \in \tilde{F}_i\ |\ \LM{f} \not\in \LM{F}\}$\;
  \For{$h \in \tilde{F}^+_{i}$}{
    $P \longleftarrow P \bigcup \{\textsc{Pair}(f,h): \forall f \in G\}$\;
    add $h$ to $G$\;
  }
}
\Return{$G$}\;
}
\caption{$F_{4}$ (simplified)} 
\label{alg:f4}
\end{algorithm}


\section{Relationship between the MXL Algorithms and $F_4$} \label{sec:relation}
In this section we discuss the relation between MXL$_3$ and $F_4$.  It was shown in \cite{DBLP:conf/asiacrypt/ArsFIKS04} that XL can be understood as a redundant variant of $F_4$ (cf.\ Lemma~\ref{lem:xl}).  Thus, we know that the ``framework'' of MXL$_3$ is compatible with $F_4$. In particular, we know that in each iteration of the main loop XL will not compute any non-redundant polynomials not computed by $F_4$. Thus in order to study the connection between the two algorithms, we only have to consider the modifications made in MXL$_3$ compared to XL. That is, we consider each of these modifications independently and argue that these still perform the same useful computations as the $F_4$ algorithm.

\subsection{Mutants}
The most visible change to XL in MXL$_3$ is the special treatment given to mutants, i.e.\ when $Mu\neq\varnothing$. That is, instead of increasing the degree $d$ in each iteration, if there is a fall of degree, 
then these new elements are treated at the current or perhaps a smaller degree before the algorithm proceeds to increase the degree as normally. Thus, compared to XL, the MXL family of algorithms may terminate at a lower degree. 

On the other hand, the $F_4$ algorithm does not specify how to choose polynomials in each iteration of the main loop. Instead, the user passes a function \textsc{Sel} which specifies how to select
pairs of polynomials. 
However, in \cite{F4} it is suggested to choose the normal selection strategy \cite[p. 225]{Becker1991} for most inputs. We recall here how the normal strategy has been adopted in $F_4$.   

\begin{definition}[Normal Strategy]
Let $F=\{f_0,\dots,f_{m-1} \}$.
We shall say that a pair $(f_i,f_j) \in F \times F$ with $f_i \ne f_j$ is a critical pair. Let then $\mathcal{P} \subset F \times F $ be the set of critical pairs. We denote by $\LCM{p_{ij}}$ the least common multiple of the leading monomials of the critical pair $p_{ij} = (f_i,f_j) \in \mathcal{P}$. We also call $\deg(\LCM{p_{ij}})$ the degree of the critical pair $p_{ij}$. Further, let $$d = \min\{\deg(\LCM{p}) \mid  p \in \mathcal{P}\}$$ be the minimal degree of those least common multiples of $p$ in $\mathcal{P}$. Then the normal selection strategy selects the subset 
$$
\mathcal{P}' = \{ p \in \mathcal{P} \mid \deg(\LCM{p})=d\}.
$$
\end{definition}

We can now state our main result.
\begin{theorem}
\label{theorem:main}
Let both MXL$_3$ and $F_4$ compute a Gr\"obner basis with respect to the same degree compatible ordering on the same input. Assume that until iteration $i$ (inclusive) of the main loop both $F_4$ and MXL$_3$ computed the same list of polynomials except for redundant polynomials, i.e., the leading monomials appearing in $F_4$ divide the leading monomials appearing in MXL$_3$. Furthermore, assume that $Mu \neq \varnothing$ in Algorithm~\ref{alg:mutantxl3} at line \ref{mxl3:mutants_found} and define $k$ to be the minimal degree of a polynomial in $Mu$. The set of polynomials $F_{\leq k+1}$ considered by MXL$_3$ in the next iteration of the main loop is a {\it superset} of the polynomials considered by $F_4$ when using the \emph{Normal Selection Strategy} in the next iteration $i+1$. Furthermore, every polynomial in $F_{\leq k+1}$ not in the set considered by $F_4$ is {\it redundant} in this iteration.
\end{theorem}

\begin{proof}
We note that it follows from Corollary~\ref{corollary:iteration} that the first assumption of the theorem will be satisfied while $Mu = \varnothing$. Now assume we have $Mu \neq \varnothing$.
First consider the $F_4$ algorithm, and let \textsc{Sel} be the Normal Selection Strategy. Then, the set $\mathcal{P}_{i+1}$ will contain the S-polynomials of lowest degree in $\mathcal{P}$. Every S-polynomial in $\mathcal{P}_{i+1}$ will have at least degree $k+1$, since the set $Mu_{=k}$ is in row echelon form and $k$ is the minimal degree in $Mu$. If there exists an S-poly\-nomial of degree $k+1$ then it is of the form $t_i f_i - t_j f_j$ with $\deg(t_i f_i) = k+1$ and $\deg(t_j f_j) = k+1$, where at least one of $t_i,t_j$ has degree 1. MXL$_3$, on the other hand, constructs all multiples $t_{ij} f_i$ with $\deg(t_{ij}) = 1$ if $\deg(f_i) = k$. Furthermore, it considers all elements of degree $k+1$ in the next iteration which covers the case that one of $t_i,t_j$ is $1$.  Hence, both components of the S-polynomial are included in $F_{\leq k+1}$.

In the \emph{Symbolic Preprocessing} phase $F_4$ also constructs all components of \emph{potential} S-polynomials that could arise during the elimination. These are always of the form $f_i - t_jf_j$ where $\deg(f_i) = \deg(t_jf_j)$. Since MXL$_3$ considers all monomial multiplies of all $f_j$ up to degree $k+1$ in the next iteration, these components are also included in the set $F_{k+1}$.

Recall from Corollary~\ref{corollary:cancel_general} that all $f = \Sigma^{t-1}_{i=0} c_ix^{\alpha(i)}f_i$ can be rewritten as $$f = \sum_{j=0}^{t-2} b_{j}x^{\delta-\tau_{j}}S(f_j,f_{j+1})$$ if $f < \max\{x^{\alpha(i)}f_i\}$. Note that $\deg(x^\delta) \leq k+1$ for $F_{\leq k+1}$ and that $\deg(x^{\tau_{j}}) = k + 1$ for all S-polynomials contained in $F_{\leq k+1}$. It follows that $\deg(x^{\delta - \tau_{j}}) = 0$ if $b_{j} \neq 0$. That is, any $f$ with a smaller leading term than its representation $\Sigma^{t-1}_{i=0} c_ix^{\alpha(i)}f_i$ can be computed by an $\F$-linear combination of S-polynomials: $f = \sum_{j=0}^{t-2} b_{j}S(f_j,f_{j+1})$.

It follows immediately from Corollary~\ref{corollary:cancel_general} that any multiple of $f_i$ which does not correspond to an S-polynomial is redundant in this iteration since it cannot lead to a drop of a leading monomial. \qed
\end{proof}

For the MXL algorithm, which only differs from XL when $Mu \neq\varnothing$, the following corollary is a direct consequence of Theorem~\ref{theorem:main} and Corollary~\ref{corollary:iteration}.

\begin{corollary}
MXL can be simulated using  $F_4$  (described in Algorithm~\ref{alg:f4}) by adding redundant pairs and using the \emph{Normal Selection Strategy}.
\end{corollary}

We note however that the MXL$_3$ algorithm may improve upon MXL when $Mu = \varnothing$ by using a ``partial enlargement'' strategy, which we discuss below.

\subsection{Partial Enlargement}
\label{sec:partion}
The ``partial enlargement'' technique was introduced in MXL$_2$ and is also applied in MXL$_3$. Instead of multiplying every polynomial $f_i \in F$ by all variables $x_0,\ldots,x_{n-1}$ only a subset $\LV{F,x}$ is considered. This subset is increased in each iteration by increasing $x$. In the language of linear algebra, the algorithm first computes the row echelon form of a submatrix in the lower right corner. If that does not suffice to produce elements of smaller degree, a larger submatrix is considered.

This corresponds to selecting a subset of S-polynomials with small least common multiple in $\textsc{Sel}$ instead of selecting all polynomials of minimal degree. We note that both the \textsc{PolyBoRi} package \cite{polybori} and \textsc{Magma} computer algebra system \cite{magma} accept an option to restrict the number of S-polynomials considered in each iteration. However, the strategy for how the number of S-polynomials is chosen in \textsc{Magma} and \textsc{PolyBoRi} is different from MXL$_3$. In the former ones, a \emph{constant} number of S-polynomials is chosen as specified by the user; in the latter (MXL$_3$) a 
\emph{changeable} number of S-polynomials is chosen based on the partition by leading variable. The strategy employed in MXL$_3$ will consider S-polynomials $S(f,g)$ where both $f$ and $g$ have leading variable at most $x$ (inclusive). That is, if there is an S-polynomial $S(f,g) = t_f\cdot f - t_g\cdot  g$ with $\LV{f} < \LV{g}$, MXL$_3$ will construct $t_f \cdot f$ when considering $\LV{F,\LV{f}}$ and $t_g \cdot g$ when considering $\LV{F, \LV{g}}$. Since $F_{\leq d}$ contains all elements of degree at most $d$, both components are included in the matrix when $\LV{F,\LV{g}}$ are considered. 

It is currently not clear which strategy for selecting subsets of S-polynomials is beneficial under which conditions. It should be noted however that if the size of the matrix is the main concern then selecting exactly the smallest S-polynomial in each iteration would be optimal; just as Buchberger's algorithm does. On the other hand, the contribution of algorithms such as $F_4$ is to improve performance by considering more than one S-polynomial in each iteration. Thus, it is not certain that using matrix sizes as a main measure of comparison gives an adequate picture of the performance of these algorithms.

\subsection{Termination Criterion} \label{sec:termination}
The key contribution of the MXL$_3$ algorithm is the introduction of a new criterion to detect when a Gr\"obner basis is found. Since the MXL family does not use the concept of critical pairs, standard termination criteria such as an empty list of pairs are not immediately applicable. In Lemma~\ref{lem:terminate} we give an equivalent variant  of this criterion, rephrased  to be more suitable for our discussion. 
\begin{lemma}[Proposition 3 in \cite{mxl3}] \label{lem:terminate}
Let $G=\{g_0,\ldots, g_{s-1}\}$ be a finite subset of $\FX$ with $D$ being the highest degree of its elements. Suppose that the following hold:
\begin{enumerate}
 \item all monomials of degree $D$ in $\FX$ are divisible by a leading monomial of some $g_i \in G$; and
 \item if $H = G \cup \{t \cdot g_i \mid g_i \in G, t \textnormal{ a monomial and } \deg(t \cdot g_i) \leq D + 1\}$, there exists $\tilde{H}$ -- a row echelon form of $H$ -- such that $\LM{\tilde{H}_{\leq D}} \subset \ideal{\LM{G}}$.
\end{enumerate}
Then $G$ is a Gr\"obner basis.
\end{lemma}
Note that condition 1 implies that the ideal generated by $G$ is 0-dimensional.

The MXL$_3$ algorithm uses a termination criterion based on Lemma~\ref{lem:terminate} and thus will consider matrices up to degree $D+1$ (where $D$ is defined as in Lemma~\ref{lem:terminate}). The $F_4$ algorithm, on the other hand, will terminate once the list of critical pairs is empty. It is obvious that no new pairs will be created after the Gr\"obner basis is found, since all reductions will lead 
to zero in this situation. However, if we consider $F_4$ as given in Algorithm~\ref{alg:f4}, one can see that the algorithm may consider pairs of degree $>D+1$ after a Gr\"obner basis is discovered, if those pairs were constructed before the Gr\"obner basis is found. Put differently, the simplified $F_4$ variant considered in this work does not prune the list of critical pairs based on the current basis $G$. However, the {\em full} $F_4$ algorithm as specified in \cite[p. 69]{F4} does indeed prune the list $P$ by calling a subroutine called \textsc{Update}. In \cite{F4} a reference to \cite[p. 230]{Becker1991} is made -- which applies Buchberger's first and second criteria using the Gebauer-M\"oller installation -- as an example of such a routine. 

The question thus becomes whether Buchberger's first and/or second criterion will remove all pairs of degree $>D+1$ if the conditions (1) and (2) of Lemma~\ref{lem:terminate} hold. An algorithmic variant of Buchberger's second criterion is given in the Lemma below.
\begin{lemma}[Buchberger's second criterion]
Let $p, g_1,g_2 \in  \FX$  be
such that $$\LM{p} \mid  \LCM{ \LM{g_1}, \LM{g_2}}.$$ and $S(g_1, p)$, $S(g_2,p)$ have already been considered. Then $S(g_1, g_2)$ does not need to be considered and can be discarded.
\end{lemma}
We can now prove that the full $F_4$ algorithm will not consider pairs of higher degree than the MXL$_3$ 
when applying Buchberger's second criterion.
\begin{proposition}
We assume a degree compatible ordering on $\FX$.
If during a Gr\"obner basis computation using the full $F_4$ algorithm 
conditions (1) and (2) of Lemma~\ref{lem:terminate} hold, then Buchberger's second criterion will remove any pair of degree $> D+1$ from the list of critical pairs. As a result $F_4$ will consider critical pairs of degree at most $D+1$.
\end{proposition}
Our proof follows very closely the original proof of Lemma~\ref{lem:terminate} in \cite{mxl3}.
\begin{proof}
Let $G=\{g_0,\ldots, g_{s-1}\}$ be a finite subset of $\FX$ with $D$ being the highest degree of its elements such that: 
\begin{enumerate}
 \item all monomials of degree $D$ in $\FX$ are divisible by a leading monomial of some $g_i \in G$; and
 \item if $H = G \cup \{t \cdot g_i \mid g_i \in G, t \textnormal{ a monomial and } \deg(t \cdot g_i) \leq D + 1\}$, there exists $\tilde{H}$ -- a row echelon form of $H$ -- such that $\LM{\tilde{H}_{\leq D}} \subset \ideal{\LM{G}}$.
\end{enumerate}
We denote the S-polynomial $S(g_i,g_j)$ by $f$, and let $d={\rm deg}(f)$. We only have to consider pairs of degree $d>D+1$. 

To do so, let $m=\LCM{\LM{g_i}, \LM{g_j}}$. There exist monomials $m_i, m_j$ such that $m = m_i \cdot \LM{g_i} = m_j \cdot \LM{g_j}$. 
It is clear that $\textsc{GCD}(m_i,m_j) = 1$.

By assumption ${\rm deg}(g_i)$ and ${\rm deg}(g_j)$ are at most equal to $D$. This implies that ${\rm deg}(m_j) \geq 2$ (resp. ${\rm deg}(m_j) \geq 2$)
since $d>D+1$. It is then possible to write $m_i=m_{i,1}\cdot m_{i,2}$ such that  ${\rm deg}(g_i) +{\rm deg}(m_{i,2})=D+1$ 
and  ${\rm deg}(m_{i,1}) \geq 1$.
A similar decomposition can be found for $m_j=m_{j,1}\cdot m_{j,2}$.         
Thus, we have that all monomials $m_{i,1}, m_{i,2}, m_{j,1}\cdot m_{j,2}$ are of degree $\geq 1$. 

Now, let $m^*= \frac{m}{m_{i,1}\cdot m_{j,1}}$. By construction, we have $$\LCM{m^*, \LM{g_i}}=m/m_{i,1} \textnormal{ (resp. } \LCM{m^*, \LM{g_j}}=m/m_{j,1}\textnormal{)},$$ which divides $m$ properly.  
We also have ${\rm deg}(m^*) \leq D$. Since $m_1$ and $m_2$ must be distinct, we have that $m^*$ cannot be equal to either $\LM{g_i}$ or $\LM{g_j}$. By condition $1$, there exists $g \in G \setminus \{g_1, g_2 \}$ such that  with $\LM{g}= m^*$.  In addition $$\deg(\LCM{\LM{g}, \LM{g_i}} < \deg(m)$$ and $\deg(\LCM{\LM{g}, \LM{g_j}} < \deg(m).$ Thus, $S(g,g_i)$ and $S(g,g_j)$ are being considered at a lower degree than $D+1$.

Finally, $m^*$ divides $m=\LCM{\LM{g_i}, \LM{g_j}}$ by construction. It then follows from Buchberger's second criterion  that $f=S(g_i,g_j)$ does not need to be considered and is discarded.
\qed
\end{proof}


\section{Conclusion}
\label{sec:conclusion}
In this work we have studied the MXL family of algorithms, and their connections to Gr\"obner bases theory. We demonstrated that the mutant strategy as used in the MXL algorithms is in fact a 
redundant variant of the Normal Selection Strategy. 
Furthermore, we showed that the partial enlargement strategy proposed in \cite{mxl2} corresponds to selecting a subset of S-polynomials of minimal degree in each iteration of algorithms such as $F_4$. 
As a result, we conclude that both the MXL and MXL$_2$ algorithms can be seen as redundant variants of the $F_4$ algorithm, although the latter may select critical pairs differently from usual $F_4$ implementations. Finally, we studied the novel termination criterion proposed in \cite{mxl3} and concluded that it does not allow the algorithm to terminate at a lower degree than $F_4$. Consequently, we conclude that MXL$_3$ too can be understood as a redundant variant of the $F_4$ algorithm. However, here too we emphasise that it might selects S-polynomials differently from standard $F_4$ implementations due to the partial enlargement strategy.

We conclude with a brief discussion on what we view as the limitations of using running times as the basis for comparison between Gr\"obner basis {\it algorithms}. Linear algebra-based Gr\"obner bases algorithms allow several degrees of freedom to the designer and implementer of the algorithm to generate the matrices, and selection of strategies can drastically affect the efficiency of the computations. Furthermore, the specific implementation details and sub-algorithms used in the implementation (e.g., the package used for performing the Gaussian reductions, the internal representation of sparse matrices, etc.) will also have great effect on running times and memory requirements (cf., \ref{app:timings} for an example).

In fact, we claim that three almost-independent aspects will affect running times of such algorithms: the mathematical details of the algorithm itself, the strategies and heuristics used in the implementation, and the
low-level implementation details. The first aspect was the main focus of interest in this paper, but it should be clear that 
our results do not preclude that particular {\it implementations} of MutantXL algorithms can outperform particular {\it implementations} of $F_4/F_5$ in some situations. On the other hand, we are aware that it is difficult to compare the complexity of Gr\"obner basis algorithms and strategies and that designers often have little choice but to resort to experimental data to demonstrate the viability of their approach.

\section{Acknowledgements}
The work described in this paper has been supported by the Royal Society grant JP090728 and by the Commission of the European
Communities through the ICT program under contract ICT-2007-216676 (ECRYPT-II). Martin R. Albrecht,
Jean-Charles Faugere, and Ludovic Perret are also supported by the French ANR under the Computer Algebra
and Cryptography (CAC) project (ANR-09-JCJCJ-0064-01) and the EXACTA project (ANR-09-BLAN-0371-01).
We would like to thank Stanislav Bulygin, Jintai Ding and Mohamed Saied Emam Mohamed for helpful comments and discusssions on earlier drafts of this work.

\clearpage

\bibliographystyle{elsart-harv}
\bibliography{mutant}

\clearpage
\appendix

\section{Effect of Linear Algebra Implementations on Gr{\"o}bner Basis Computations}
\label{app:timings}
To show the effect of the linear algebra implementation, we compare two implementations of the $F_4$ algorithm. The only difference is the linear algebra package used to perform the Gaussian elimination step. We compare the original FGb implementation with the new linear algebra package described in~\cite{FL10b}. However, to make the comparison fair we only use a sequential version of the package described in~\cite{FL10b}. To compare, we consider the  reduction  of the $7$th matrix occurring in the computation of a Gr\"obner basis of the standard benchmark Katsura $12$ over $\F_{65521}$, as well as the full Gr\"obner basis computation.  Typically, it takes 326.1 sec and 250 Mbytes to reduce the $7$th matrix with FGb and $83.7$ seconds and $682$ Mbytes using FGb with the library from \cite{FL10b}.
\begin{table}[ht]
\begin{center}
\begin{scriptsize}
\caption{Algorithm: F4 -- Katsura $14$ over $\F_{65521}$.}
  \begin{tabular}{lcc}
    \hline
                        &       Matrix 7 ($21,915 \times 23,127$)       &  Full Gr\"obner basis \\
FGb/CPU         &               83 s.           &       326 s. \\
FGb/Memory              &               250 Mbytes              &       262 Mbytes\\
FGb/Pasco/CPU  \cite{FL10b} (1 core) &          32 s.   &               151 s.\\
FGb/Pasco/Memory \cite{FL10b}                     &     682 Mbytes        &              682 Mbytes\\
    \hline
\end{tabular}
\end{scriptsize}
\end{center}
\end{table}


\end{document}
